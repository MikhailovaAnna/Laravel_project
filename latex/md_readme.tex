\subsection*{Рабработка программного обеспечения для поэтапного вывода части большой таблицы БД в W\+E\+B-\/интерфейсе пользователя}

ПО предназначено для отображения таблицы БД, информация из которой отображается поэтапно, по запросу пользователя следующего или предыдущего блока данных, или блока, начинающегося с определенной даты, создан поиск данных в БД по дате, сообщению и номеру телефона.

Разработка велась с использованием C\+MF \hyperlink{namespaceLaravel}{Laravel}, на языках программирования P\+HP 7.\+3, H\+T\+ML, C\+SS.

Таблица содержит столбцы\+: \char`\"{}Номер телефона\char`\"{}, \char`\"{}Дата\char`\"{}, \char`\"{}Время\char`\"{}, \char`\"{}Текст сообщения\char`\"{}.

\subsection*{Инструкция}


\begin{DoxyItemize}
\item Установка Composer
\end{DoxyItemize}

Composer неободим для предоставления средств по управлению зависимостями в P\+H\+P-\/приложении. \begin{DoxyVerb}sudo apt install composer
\end{DoxyVerb}



\begin{DoxyItemize}
\item Установка \hyperlink{namespaceLaravel}{Laravel} с помощью Composer
\end{DoxyItemize}


\begin{DoxyCode}
composer global require laravel/installer
\end{DoxyCode}


Изменяем переменную P\+A\+TH\+: \begin{DoxyVerb}export PATH="$HOME/.config/composer/vendor/bin:$PATH"
\end{DoxyVerb}



\begin{DoxyItemize}
\item Создание проекта \hyperlink{namespaceLaravel}{Laravel}\+: 
\begin{DoxyCode}
laravel new new\_laravel
\end{DoxyCode}
 где new\+\_\+laravel -\/ это имя проекта.
\end{DoxyItemize}

Можно проверить работу проекта, запустив сервер командой\+: \begin{DoxyVerb}php artisan serve
\end{DoxyVerb}


Если увидела стартовую страницу \hyperlink{namespaceLaravel}{Laravel} -\/ все в порядке!


\begin{DoxyItemize}
\item Установка Postgre\+S\+QL 
\begin{DoxyCode}
sudo apt-get update
sudo apt-get install postgresql-10.4
\end{DoxyCode}

\item Создание базы данных 
\begin{DoxyCode}
sudo -u postgres psql -c "CREATE DATABASE db;"
\end{DoxyCode}
 где db -\/ название базы данных.
\end{DoxyItemize}

Запустить и отключить сервер Postgre\+S\+QL можно командами\+: \begin{DoxyVerb}sudo service postgresql start
sudo service postgresql stop
\end{DoxyVerb}



\begin{DoxyItemize}
\item Подключение проекта к БД
\end{DoxyItemize}

В файле проекта .env заменяются следующие строчки\+: \begin{DoxyVerb}DB_CONNECTION=pgsql
DB_HOST=127.0.0.1
DB_PORT=5432
DB_DATABASE=ann
DB_USERNAME=ann
DB_PASSWORD=789
\end{DoxyVerb}


D\+B\+\_\+\+D\+A\+T\+A\+B\+A\+SE, D\+B\+\_\+\+U\+S\+E\+R\+N\+A\+ME, D\+B\+\_\+\+P\+A\+S\+S\+W\+O\+RD меняем в зависимости от созданных нами БД, пользователя и пароля.

В файле config/database.\+php заменяем\+: \begin{DoxyVerb}'default' => env('DB_CONNECTION', 'pgsql'),
\end{DoxyVerb}


Перезагружаем сервер БД, и наш проект должен быть подключен к БД!


\begin{DoxyItemize}
\item Генерация тестовых данных
\end{DoxyItemize}

С помощью миграций создаем таблицу с данными, в моем случае -\/ books. \begin{DoxyVerb}php artisan make:migration books
\end{DoxyVerb}


В папке database/migrations должна появится соответствующая миграция. В ней заменяем существующую функцию up() на\+: \begin{DoxyVerb}public function up()
    {
        Schema::create('books', function (Blueprint $table) {
            $table->bigIncrements('id');
            $table->string('Phone');
            $table->text('Text');
            $table->date('Date');
            $table->timestamps();
        });
    }
\end{DoxyVerb}


В папке database/factories создаем файл Book\+Model\+Factory.\+php, с содержанием\+: \begin{DoxyVerb}<?php

/**\end{DoxyVerb}
 